% Options for packages loaded elsewhere
\PassOptionsToPackage{unicode}{hyperref}
\PassOptionsToPackage{hyphens}{url}
%
\documentclass[
]{article}
\usepackage{amsmath,amssymb}
\usepackage{iftex}
\ifPDFTeX
  \usepackage[T1]{fontenc}
  \usepackage[utf8]{inputenc}
  \usepackage{textcomp} % provide euro and other symbols
\else % if luatex or xetex
  \usepackage{unicode-math} % this also loads fontspec
  \defaultfontfeatures{Scale=MatchLowercase}
  \defaultfontfeatures[\rmfamily]{Ligatures=TeX,Scale=1}
\fi
\usepackage{lmodern}
\ifPDFTeX\else
  % xetex/luatex font selection
\fi
% Use upquote if available, for straight quotes in verbatim environments
\IfFileExists{upquote.sty}{\usepackage{upquote}}{}
\IfFileExists{microtype.sty}{% use microtype if available
  \usepackage[]{microtype}
  \UseMicrotypeSet[protrusion]{basicmath} % disable protrusion for tt fonts
}{}
\makeatletter
\@ifundefined{KOMAClassName}{% if non-KOMA class
  \IfFileExists{parskip.sty}{%
    \usepackage{parskip}
  }{% else
    \setlength{\parindent}{0pt}
    \setlength{\parskip}{6pt plus 2pt minus 1pt}}
}{% if KOMA class
  \KOMAoptions{parskip=half}}
\makeatother
\usepackage{xcolor}
\usepackage[margin=1in]{geometry}
\usepackage{color}
\usepackage{fancyvrb}
\newcommand{\VerbBar}{|}
\newcommand{\VERB}{\Verb[commandchars=\\\{\}]}
\DefineVerbatimEnvironment{Highlighting}{Verbatim}{commandchars=\\\{\}}
% Add ',fontsize=\small' for more characters per line
\usepackage{framed}
\definecolor{shadecolor}{RGB}{248,248,248}
\newenvironment{Shaded}{\begin{snugshade}}{\end{snugshade}}
\newcommand{\AlertTok}[1]{\textcolor[rgb]{0.94,0.16,0.16}{#1}}
\newcommand{\AnnotationTok}[1]{\textcolor[rgb]{0.56,0.35,0.01}{\textbf{\textit{#1}}}}
\newcommand{\AttributeTok}[1]{\textcolor[rgb]{0.13,0.29,0.53}{#1}}
\newcommand{\BaseNTok}[1]{\textcolor[rgb]{0.00,0.00,0.81}{#1}}
\newcommand{\BuiltInTok}[1]{#1}
\newcommand{\CharTok}[1]{\textcolor[rgb]{0.31,0.60,0.02}{#1}}
\newcommand{\CommentTok}[1]{\textcolor[rgb]{0.56,0.35,0.01}{\textit{#1}}}
\newcommand{\CommentVarTok}[1]{\textcolor[rgb]{0.56,0.35,0.01}{\textbf{\textit{#1}}}}
\newcommand{\ConstantTok}[1]{\textcolor[rgb]{0.56,0.35,0.01}{#1}}
\newcommand{\ControlFlowTok}[1]{\textcolor[rgb]{0.13,0.29,0.53}{\textbf{#1}}}
\newcommand{\DataTypeTok}[1]{\textcolor[rgb]{0.13,0.29,0.53}{#1}}
\newcommand{\DecValTok}[1]{\textcolor[rgb]{0.00,0.00,0.81}{#1}}
\newcommand{\DocumentationTok}[1]{\textcolor[rgb]{0.56,0.35,0.01}{\textbf{\textit{#1}}}}
\newcommand{\ErrorTok}[1]{\textcolor[rgb]{0.64,0.00,0.00}{\textbf{#1}}}
\newcommand{\ExtensionTok}[1]{#1}
\newcommand{\FloatTok}[1]{\textcolor[rgb]{0.00,0.00,0.81}{#1}}
\newcommand{\FunctionTok}[1]{\textcolor[rgb]{0.13,0.29,0.53}{\textbf{#1}}}
\newcommand{\ImportTok}[1]{#1}
\newcommand{\InformationTok}[1]{\textcolor[rgb]{0.56,0.35,0.01}{\textbf{\textit{#1}}}}
\newcommand{\KeywordTok}[1]{\textcolor[rgb]{0.13,0.29,0.53}{\textbf{#1}}}
\newcommand{\NormalTok}[1]{#1}
\newcommand{\OperatorTok}[1]{\textcolor[rgb]{0.81,0.36,0.00}{\textbf{#1}}}
\newcommand{\OtherTok}[1]{\textcolor[rgb]{0.56,0.35,0.01}{#1}}
\newcommand{\PreprocessorTok}[1]{\textcolor[rgb]{0.56,0.35,0.01}{\textit{#1}}}
\newcommand{\RegionMarkerTok}[1]{#1}
\newcommand{\SpecialCharTok}[1]{\textcolor[rgb]{0.81,0.36,0.00}{\textbf{#1}}}
\newcommand{\SpecialStringTok}[1]{\textcolor[rgb]{0.31,0.60,0.02}{#1}}
\newcommand{\StringTok}[1]{\textcolor[rgb]{0.31,0.60,0.02}{#1}}
\newcommand{\VariableTok}[1]{\textcolor[rgb]{0.00,0.00,0.00}{#1}}
\newcommand{\VerbatimStringTok}[1]{\textcolor[rgb]{0.31,0.60,0.02}{#1}}
\newcommand{\WarningTok}[1]{\textcolor[rgb]{0.56,0.35,0.01}{\textbf{\textit{#1}}}}
\usepackage{graphicx}
\makeatletter
\def\maxwidth{\ifdim\Gin@nat@width>\linewidth\linewidth\else\Gin@nat@width\fi}
\def\maxheight{\ifdim\Gin@nat@height>\textheight\textheight\else\Gin@nat@height\fi}
\makeatother
% Scale images if necessary, so that they will not overflow the page
% margins by default, and it is still possible to overwrite the defaults
% using explicit options in \includegraphics[width, height, ...]{}
\setkeys{Gin}{width=\maxwidth,height=\maxheight,keepaspectratio}
% Set default figure placement to htbp
\makeatletter
\def\fps@figure{htbp}
\makeatother
\setlength{\emergencystretch}{3em} % prevent overfull lines
\providecommand{\tightlist}{%
  \setlength{\itemsep}{0pt}\setlength{\parskip}{0pt}}
\setcounter{secnumdepth}{-\maxdimen} % remove section numbering
\usepackage{booktabs}
\usepackage{longtable}
\usepackage{array}
\usepackage{multirow}
\usepackage{wrapfig}
\usepackage{float}
\usepackage{colortbl}
\usepackage{pdflscape}
\usepackage{tabu}
\usepackage{threeparttable}
\usepackage{threeparttablex}
\usepackage[normalem]{ulem}
\usepackage{makecell}
\usepackage{xcolor}
\ifLuaTeX
  \usepackage{selnolig}  % disable illegal ligatures
\fi
\IfFileExists{bookmark.sty}{\usepackage{bookmark}}{\usepackage{hyperref}}
\IfFileExists{xurl.sty}{\usepackage{xurl}}{} % add URL line breaks if available
\urlstyle{same}
\hypersetup{
  pdftitle={Homework Metrics 2b: Federal Reserve's interest rates and global outstanding credit. A univariate and multivariates analysis},
  pdfauthor={Valentin Auplat, Nino Laffray},
  hidelinks,
  pdfcreator={LaTeX via pandoc}}

\title{Homework Metrics 2b: Federal Reserve's interest rates and global
outstanding credit. A univariate and multivariates analysis}
\author{Valentin Auplat, Nino Laffray}
\date{May 2025}

\begin{document}
\maketitle

\hypertarget{question-1-plots-of-the-variables-of-interest.}{%
\section{Question 1: Plots of the variables of
interest.}\label{question-1-plots-of-the-variables-of-interest.}}

\includegraphics{HW-Metrics-2b-Valentin-Auplat-Nino-Laffray_files/figure-latex/unnamed-chunk-5-1.pdf}
\includegraphics{HW-Metrics-2b-Valentin-Auplat-Nino-Laffray_files/figure-latex/unnamed-chunk-5-2.pdf}
\includegraphics{HW-Metrics-2b-Valentin-Auplat-Nino-Laffray_files/figure-latex/unnamed-chunk-5-3.pdf}
Stationarity cannot be assumed from these graphs. We see that variables
present trends, but not cycles. For this first exercise, we are mostly
interested in the interest rate of the Federal Reserve and the amount of
credit borrowed outside the United States denominated in USD.

\hypertarget{question-2-stationarity-tests.}{%
\section{Question 2: Stationarity
tests.}\label{question-2-stationarity-tests.}}

\begin{verbatim}
## 
## ############################################### 
## # Augmented Dickey-Fuller Test Unit Root Test # 
## ############################################### 
## 
## Test regression trend 
## 
## 
## Call:
## lm(formula = z.diff ~ z.lag.1 + 1 + tt + z.diff.lag)
## 
## Residuals:
##      Min       1Q   Median       3Q      Max 
## -1.06709 -0.09487 -0.04901  0.13641  0.78742 
## 
## Coefficients:
##              Estimate Std. Error t value Pr(>|t|)    
## (Intercept)  0.221652   0.100899   2.197 0.031213 *  
## z.lag.1     -0.087199   0.021263  -4.101 0.000106 ***
## tt          -0.002419   0.001702  -1.421 0.159586    
## z.diff.lag1  0.543911   0.107517   5.059 3.04e-06 ***
## z.diff.lag2  0.228576   0.111798   2.045 0.044508 *  
## ---
## Signif. codes:  0 '***' 0.001 '**' 0.01 '*' 0.05 '.' 0.1 ' ' 1
## 
## Residual standard error: 0.2755 on 73 degrees of freedom
## Multiple R-squared:  0.5875, Adjusted R-squared:  0.5649 
## F-statistic: 25.99 on 4 and 73 DF,  p-value: 2.066e-13
## 
## 
## Value of test-statistic is: -4.101 6.1243 8.8495 
## 
## Critical values for test statistics: 
##       1pct  5pct 10pct
## tau3 -4.04 -3.45 -3.15
## phi2  6.50  4.88  4.16
## phi3  8.73  6.49  5.47
\end{verbatim}

\begingroup\fontsize{10}{12}\selectfont

\begin{longtable}[t]{llc}
\caption{\label{tab:unnamed-chunk-6}Augmented Dickey-Fuller Test Results}\\
\toprule
 & Statistic & Value\\
\midrule
 & Test Statistic & -4.101\\
1pct & Critical Value (1\%) & -4.040\\
5pct & Critical Value (5\%) & -3.450\\
10pct & Critical Value (10\%) & -3.150\\
\bottomrule
\end{longtable}
\endgroup{}

Although the visual plot of the FED rate shows distinct level shifts and
prolonged flat regimes (e.g., 2008--2016), the ADF test result shows
that the process is stationary once we control for trend and include
appropriate lags.

This means the visual ``non-stationarity'' may be deterministic trend or
policy-driven shifts, but not a stochastic unit root.

\begin{Shaded}
\begin{Highlighting}[]
\CommentTok{\# 2.1 ADF Test (Augmented Dickey{-}Fuller)}
\NormalTok{adf\_test }\OtherTok{\textless{}{-}} \FunctionTok{ur.df}\NormalTok{(df\_long[,}\FunctionTok{c}\NormalTok{(}\StringTok{"Borrowers outside United States"}\NormalTok{)], }\AttributeTok{type =} \StringTok{"trend"}\NormalTok{, }\AttributeTok{lags =} \DecValTok{4}\NormalTok{)  }\CommentTok{\# Try "drift" or "none" too}
\FunctionTok{summary}\NormalTok{(adf\_test)}
\end{Highlighting}
\end{Shaded}

\begin{verbatim}
## 
## ############################################### 
## # Augmented Dickey-Fuller Test Unit Root Test # 
## ############################################### 
## 
## Test regression trend 
## 
## 
## Call:
## lm(formula = z.diff ~ z.lag.1 + 1 + tt + z.diff.lag)
## 
## Residuals:
##     Min      1Q  Median      3Q     Max 
## -351.00  -51.74   -5.20   53.39  280.03 
## 
## Coefficients:
##              Estimate Std. Error t value Pr(>|t|)    
## (Intercept) 195.92769   50.14905   3.907 0.000215 ***
## z.lag.1      -0.15726    0.04373  -3.596 0.000603 ***
## tt           22.90499    6.11895   3.743 0.000372 ***
## z.diff.lag1   0.12606    0.11464   1.100 0.275299    
## z.diff.lag2   0.15902    0.11537   1.378 0.172561    
## z.diff.lag3  -0.10185    0.11585  -0.879 0.382350    
## z.diff.lag4   0.15692    0.11484   1.366 0.176243    
## ---
## Signif. codes:  0 '***' 0.001 '**' 0.01 '*' 0.05 '.' 0.1 ' ' 1
## 
## Residual standard error: 105.6 on 69 degrees of freedom
## Multiple R-squared:  0.2617, Adjusted R-squared:  0.1975 
## F-statistic: 4.076 on 6 and 69 DF,  p-value: 0.001478
## 
## 
## Value of test-statistic is: -3.5959 8.5744 7.9044 
## 
## Critical values for test statistics: 
##       1pct  5pct 10pct
## tau3 -4.04 -3.45 -3.15
## phi2  6.50  4.88  4.16
## phi3  8.73  6.49  5.47
\end{verbatim}

Idem.

\begin{Shaded}
\begin{Highlighting}[]
\CommentTok{\# 2.3 KPSS Test (Stationarity)}
\NormalTok{kpss\_test }\OtherTok{\textless{}{-}} \FunctionTok{ur.kpss}\NormalTok{(df\_long[,}\FunctionTok{c}\NormalTok{(}\StringTok{"FED\_rate"}\NormalTok{)], }\AttributeTok{type =} \StringTok{"tau"}\NormalTok{, }\AttributeTok{lags =} \StringTok{"long"}\NormalTok{)}
\FunctionTok{summary}\NormalTok{(kpss\_test)}
\end{Highlighting}
\end{Shaded}

\begin{verbatim}
## 
## ####################### 
## # KPSS Unit Root Test # 
## ####################### 
## 
## Test is of type: tau with 11 lags. 
## 
## Value of test-statistic is: 0.0923 
## 
## Critical value for a significance level of: 
##                 10pct  5pct 2.5pct  1pct
## critical values 0.119 0.146  0.176 0.216
\end{verbatim}

\begin{Shaded}
\begin{Highlighting}[]
\CommentTok{\# 2.3 KPSS Test (Stationarity)}
\NormalTok{kpss\_test }\OtherTok{\textless{}{-}} \FunctionTok{ur.kpss}\NormalTok{(df\_long[,}\FunctionTok{c}\NormalTok{(}\StringTok{"Borrowers outside United States"}\NormalTok{)], }\AttributeTok{type =} \StringTok{"tau"}\NormalTok{, }\AttributeTok{lags =} \StringTok{"long"}\NormalTok{)}
\FunctionTok{summary}\NormalTok{(kpss\_test)}
\end{Highlighting}
\end{Shaded}

\begin{verbatim}
## 
## ####################### 
## # KPSS Unit Root Test # 
## ####################### 
## 
## Test is of type: tau with 11 lags. 
## 
## Value of test-statistic is: 0.1485 
## 
## Critical value for a significance level of: 
##                 10pct  5pct 2.5pct  1pct
## critical values 0.119 0.146  0.176 0.216
\end{verbatim}

\begin{Shaded}
\begin{Highlighting}[]
\FunctionTok{library}\NormalTok{(forecast)}
\FunctionTok{Acf}\NormalTok{(df\_long[,}\FunctionTok{c}\NormalTok{(}\StringTok{"FED\_rate"}\NormalTok{)], }\AttributeTok{lag.max=}\DecValTok{20}\NormalTok{) }\CommentTok{\# the MA part could be q=3, but a lot of non{-}null higher order lags}
\end{Highlighting}
\end{Shaded}

\includegraphics{HW-Metrics-2b-Valentin-Auplat-Nino-Laffray_files/figure-latex/unnamed-chunk-10-1.pdf}

\begin{Shaded}
\begin{Highlighting}[]
\FunctionTok{Pacf}\NormalTok{(df\_long[,}\FunctionTok{c}\NormalTok{(}\StringTok{"FED\_rate"}\NormalTok{)], }\AttributeTok{lag.max=}\DecValTok{20}\NormalTok{) }\CommentTok{\# the AR part could be p=3 with the full sample}
\end{Highlighting}
\end{Shaded}

\includegraphics{HW-Metrics-2b-Valentin-Auplat-Nino-Laffray_files/figure-latex/unnamed-chunk-10-2.pdf}

\begin{Shaded}
\begin{Highlighting}[]
\NormalTok{ar3 }\OtherTok{\textless{}{-}} \FunctionTok{Arima}\NormalTok{(df\_long[,}\FunctionTok{c}\NormalTok{(}\StringTok{"FED\_rate"}\NormalTok{)], }\AttributeTok{order=}\FunctionTok{c}\NormalTok{(}\DecValTok{1}\NormalTok{,}\DecValTok{2}\NormalTok{,}\DecValTok{0}\NormalTok{), }\AttributeTok{include.constant=}\ConstantTok{TRUE}\NormalTok{) }\CommentTok{\# start with the AR part, assume integration order 0 (stationary)}
\FunctionTok{summary}\NormalTok{(ar3) }\CommentTok{\# last lag and constant are significant}
\end{Highlighting}
\end{Shaded}

\begin{verbatim}
## Series: df_long[, c("FED_rate")] 
## ARIMA(1,2,0) 
## 
## Coefficients:
##           ar1
##       -0.2103
## s.e.   0.1102
## 
## sigma^2 = 0.1014:  log likelihood = -21.23
## AIC=46.46   AICc=46.62   BIC=51.2
## 
## Training set error measures:
##                       ME      RMSE       MAE      MPE     MAPE      MASE
## Training set -0.01488811 0.3125498 0.1832147 9.170396 27.47666 0.2016802
##                     ACF1
## Training set -0.01449538
\end{verbatim}

\begin{Shaded}
\begin{Highlighting}[]
\FunctionTok{plot}\NormalTok{(}\FunctionTok{cbind}\NormalTok{(ar3}\SpecialCharTok{$}\NormalTok{x, ar3}\SpecialCharTok{$}\NormalTok{fitted), }\AttributeTok{plot.type=}\StringTok{"s"}\NormalTok{, }\AttributeTok{col=}\FunctionTok{c}\NormalTok{(}\StringTok{"black"}\NormalTok{,}\StringTok{"red"}\NormalTok{), }\AttributeTok{lty=}\DecValTok{1}\NormalTok{)}
\FunctionTok{legend}\NormalTok{(}\StringTok{"topright"}\NormalTok{, }\AttributeTok{legend=}\FunctionTok{c}\NormalTok{(}\StringTok{"GDP (QoQ growth rate)"}\NormalTok{,}\StringTok{"AR(3)"}\NormalTok{) ,}\AttributeTok{col=}\FunctionTok{c}\NormalTok{(}\StringTok{"black"}\NormalTok{,}\StringTok{"red"}\NormalTok{), }\AttributeTok{lty=}\DecValTok{1}\NormalTok{, }\AttributeTok{bty=}\StringTok{"n"}\NormalTok{)}
\end{Highlighting}
\end{Shaded}

\includegraphics{HW-Metrics-2b-Valentin-Auplat-Nino-Laffray_files/figure-latex/unnamed-chunk-10-3.pdf}

\begin{Shaded}
\begin{Highlighting}[]
\NormalTok{ma3 }\OtherTok{\textless{}{-}} \FunctionTok{Arima}\NormalTok{(df\_long[,}\FunctionTok{c}\NormalTok{(}\StringTok{"FED\_rate"}\NormalTok{)], }\AttributeTok{order=}\FunctionTok{c}\NormalTok{(}\DecValTok{0}\NormalTok{,}\DecValTok{0}\NormalTok{,}\DecValTok{3}\NormalTok{), }\AttributeTok{include.constant=}\ConstantTok{TRUE}\NormalTok{) }\CommentTok{\# continue with the MA part}
\FunctionTok{summary}\NormalTok{(ma3) }\CommentTok{\# last lag and the constant are significant.}
\end{Highlighting}
\end{Shaded}

\begin{verbatim}
## Series: df_long[, c("FED_rate")] 
## ARIMA(0,0,3) with non-zero mean 
## 
## Coefficients:
##          ma1     ma2     ma3    mean
##       1.7703  1.5287  0.6303  1.8235
## s.e.  0.0961  0.1198  0.0661  0.2706
## 
## sigma^2 = 0.2653:  log likelihood = -61.03
## AIC=132.07   AICc=132.87   BIC=144.04
## 
## Training set error measures:
##                       ME      RMSE       MAE       MPE     MAPE     MASE
## Training set -0.01726954 0.5021898 0.3901893 -102.0232 126.3818 0.429515
##                   ACF1
## Training set 0.3113716
\end{verbatim}

\begin{Shaded}
\begin{Highlighting}[]
\FunctionTok{plot}\NormalTok{(}\FunctionTok{cbind}\NormalTok{(ma3}\SpecialCharTok{$}\NormalTok{x,ma3}\SpecialCharTok{$}\NormalTok{fitted), }\AttributeTok{plot.type=}\StringTok{"s"}\NormalTok{, }\AttributeTok{col=}\FunctionTok{c}\NormalTok{(}\StringTok{"black"}\NormalTok{,}\StringTok{"red"}\NormalTok{), }\AttributeTok{lty=}\DecValTok{1}\NormalTok{)}
\FunctionTok{legend}\NormalTok{(}\StringTok{"topright"}\NormalTok{, }\AttributeTok{legend=}\FunctionTok{c}\NormalTok{(}\StringTok{"GDP (QoQ growth rate)"}\NormalTok{,}\StringTok{"MA(3)"}\NormalTok{) ,}\AttributeTok{col=}\FunctionTok{c}\NormalTok{(}\StringTok{"black"}\NormalTok{,}\StringTok{"red"}\NormalTok{), }\AttributeTok{lty=}\DecValTok{1}\NormalTok{, }\AttributeTok{bty=}\StringTok{"n"}\NormalTok{)}
\end{Highlighting}
\end{Shaded}

\includegraphics{HW-Metrics-2b-Valentin-Auplat-Nino-Laffray_files/figure-latex/unnamed-chunk-10-4.pdf}

\begin{Shaded}
\begin{Highlighting}[]
\CommentTok{\# Fit AR models}
\NormalTok{ar1 }\OtherTok{\textless{}{-}} \FunctionTok{Arima}\NormalTok{(df\_long[,}\FunctionTok{c}\NormalTok{(}\StringTok{"FED\_rate"}\NormalTok{)], }\AttributeTok{order =} \FunctionTok{c}\NormalTok{(}\DecValTok{1}\NormalTok{, }\DecValTok{0}\NormalTok{, }\DecValTok{0}\NormalTok{), }\AttributeTok{include.constant =} \ConstantTok{TRUE}\NormalTok{)}
\NormalTok{ar2 }\OtherTok{\textless{}{-}} \FunctionTok{Arima}\NormalTok{(df\_long[,}\FunctionTok{c}\NormalTok{(}\StringTok{"FED\_rate"}\NormalTok{)], }\AttributeTok{order =} \FunctionTok{c}\NormalTok{(}\DecValTok{2}\NormalTok{, }\DecValTok{0}\NormalTok{, }\DecValTok{0}\NormalTok{), }\AttributeTok{include.constant =} \ConstantTok{TRUE}\NormalTok{)}
\NormalTok{ar3 }\OtherTok{\textless{}{-}} \FunctionTok{Arima}\NormalTok{(df\_long[,}\FunctionTok{c}\NormalTok{(}\StringTok{"FED\_rate"}\NormalTok{)], }\AttributeTok{order =} \FunctionTok{c}\NormalTok{(}\DecValTok{3}\NormalTok{, }\DecValTok{0}\NormalTok{, }\DecValTok{0}\NormalTok{), }\AttributeTok{include.constant =} \ConstantTok{TRUE}\NormalTok{)}

\CommentTok{\# Create comparison table}
\NormalTok{model\_comparison }\OtherTok{\textless{}{-}} \FunctionTok{data.frame}\NormalTok{(}
  \AttributeTok{Model =} \FunctionTok{c}\NormalTok{(}\StringTok{"AR(1)"}\NormalTok{, }\StringTok{"AR(2)"}\NormalTok{, }\StringTok{"AR(3)"}\NormalTok{),}
  \AttributeTok{AIC =} \FunctionTok{c}\NormalTok{(}\FunctionTok{AIC}\NormalTok{(ar1), }\FunctionTok{AIC}\NormalTok{(ar2), }\FunctionTok{AIC}\NormalTok{(ar3)),}
  \AttributeTok{BIC =} \FunctionTok{c}\NormalTok{(}\FunctionTok{BIC}\NormalTok{(ar1), }\FunctionTok{BIC}\NormalTok{(ar2), }\FunctionTok{BIC}\NormalTok{(ar3)),}
  \AttributeTok{LogLikelihood =} \FunctionTok{c}\NormalTok{(}\FunctionTok{logLik}\NormalTok{(ar1), }\FunctionTok{logLik}\NormalTok{(ar2), }\FunctionTok{logLik}\NormalTok{(ar3))}
\NormalTok{)}

\FunctionTok{print}\NormalTok{(model\_comparison)}
\end{Highlighting}
\end{Shaded}

\begin{verbatim}
##   Model      AIC      BIC LogLikelihood
## 1 AR(1) 98.45321 105.6366     -46.22661
## 2 AR(2) 40.18830  49.7661     -16.09415
## 3 AR(3) 39.16065  51.1329     -14.58033
\end{verbatim}

\end{document}
