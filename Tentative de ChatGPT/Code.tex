% Options for packages loaded elsewhere
\PassOptionsToPackage{unicode}{hyperref}
\PassOptionsToPackage{hyphens}{url}
%
\documentclass[
]{article}
\usepackage{amsmath,amssymb}
\usepackage{iftex}
\ifPDFTeX
  \usepackage[T1]{fontenc}
  \usepackage[utf8]{inputenc}
  \usepackage{textcomp} % provide euro and other symbols
\else % if luatex or xetex
  \usepackage{unicode-math} % this also loads fontspec
  \defaultfontfeatures{Scale=MatchLowercase}
  \defaultfontfeatures[\rmfamily]{Ligatures=TeX,Scale=1}
\fi
\usepackage{lmodern}
\ifPDFTeX\else
  % xetex/luatex font selection
\fi
% Use upquote if available, for straight quotes in verbatim environments
\IfFileExists{upquote.sty}{\usepackage{upquote}}{}
\IfFileExists{microtype.sty}{% use microtype if available
  \usepackage[]{microtype}
  \UseMicrotypeSet[protrusion]{basicmath} % disable protrusion for tt fonts
}{}
\makeatletter
\@ifundefined{KOMAClassName}{% if non-KOMA class
  \IfFileExists{parskip.sty}{%
    \usepackage{parskip}
  }{% else
    \setlength{\parindent}{0pt}
    \setlength{\parskip}{6pt plus 2pt minus 1pt}}
}{% if KOMA class
  \KOMAoptions{parskip=half}}
\makeatother
\usepackage{xcolor}
\usepackage[margin=1in]{geometry}
\usepackage{color}
\usepackage{fancyvrb}
\newcommand{\VerbBar}{|}
\newcommand{\VERB}{\Verb[commandchars=\\\{\}]}
\DefineVerbatimEnvironment{Highlighting}{Verbatim}{commandchars=\\\{\}}
% Add ',fontsize=\small' for more characters per line
\usepackage{framed}
\definecolor{shadecolor}{RGB}{248,248,248}
\newenvironment{Shaded}{\begin{snugshade}}{\end{snugshade}}
\newcommand{\AlertTok}[1]{\textcolor[rgb]{0.94,0.16,0.16}{#1}}
\newcommand{\AnnotationTok}[1]{\textcolor[rgb]{0.56,0.35,0.01}{\textbf{\textit{#1}}}}
\newcommand{\AttributeTok}[1]{\textcolor[rgb]{0.13,0.29,0.53}{#1}}
\newcommand{\BaseNTok}[1]{\textcolor[rgb]{0.00,0.00,0.81}{#1}}
\newcommand{\BuiltInTok}[1]{#1}
\newcommand{\CharTok}[1]{\textcolor[rgb]{0.31,0.60,0.02}{#1}}
\newcommand{\CommentTok}[1]{\textcolor[rgb]{0.56,0.35,0.01}{\textit{#1}}}
\newcommand{\CommentVarTok}[1]{\textcolor[rgb]{0.56,0.35,0.01}{\textbf{\textit{#1}}}}
\newcommand{\ConstantTok}[1]{\textcolor[rgb]{0.56,0.35,0.01}{#1}}
\newcommand{\ControlFlowTok}[1]{\textcolor[rgb]{0.13,0.29,0.53}{\textbf{#1}}}
\newcommand{\DataTypeTok}[1]{\textcolor[rgb]{0.13,0.29,0.53}{#1}}
\newcommand{\DecValTok}[1]{\textcolor[rgb]{0.00,0.00,0.81}{#1}}
\newcommand{\DocumentationTok}[1]{\textcolor[rgb]{0.56,0.35,0.01}{\textbf{\textit{#1}}}}
\newcommand{\ErrorTok}[1]{\textcolor[rgb]{0.64,0.00,0.00}{\textbf{#1}}}
\newcommand{\ExtensionTok}[1]{#1}
\newcommand{\FloatTok}[1]{\textcolor[rgb]{0.00,0.00,0.81}{#1}}
\newcommand{\FunctionTok}[1]{\textcolor[rgb]{0.13,0.29,0.53}{\textbf{#1}}}
\newcommand{\ImportTok}[1]{#1}
\newcommand{\InformationTok}[1]{\textcolor[rgb]{0.56,0.35,0.01}{\textbf{\textit{#1}}}}
\newcommand{\KeywordTok}[1]{\textcolor[rgb]{0.13,0.29,0.53}{\textbf{#1}}}
\newcommand{\NormalTok}[1]{#1}
\newcommand{\OperatorTok}[1]{\textcolor[rgb]{0.81,0.36,0.00}{\textbf{#1}}}
\newcommand{\OtherTok}[1]{\textcolor[rgb]{0.56,0.35,0.01}{#1}}
\newcommand{\PreprocessorTok}[1]{\textcolor[rgb]{0.56,0.35,0.01}{\textit{#1}}}
\newcommand{\RegionMarkerTok}[1]{#1}
\newcommand{\SpecialCharTok}[1]{\textcolor[rgb]{0.81,0.36,0.00}{\textbf{#1}}}
\newcommand{\SpecialStringTok}[1]{\textcolor[rgb]{0.31,0.60,0.02}{#1}}
\newcommand{\StringTok}[1]{\textcolor[rgb]{0.31,0.60,0.02}{#1}}
\newcommand{\VariableTok}[1]{\textcolor[rgb]{0.00,0.00,0.00}{#1}}
\newcommand{\VerbatimStringTok}[1]{\textcolor[rgb]{0.31,0.60,0.02}{#1}}
\newcommand{\WarningTok}[1]{\textcolor[rgb]{0.56,0.35,0.01}{\textbf{\textit{#1}}}}
\usepackage{graphicx}
\makeatletter
\def\maxwidth{\ifdim\Gin@nat@width>\linewidth\linewidth\else\Gin@nat@width\fi}
\def\maxheight{\ifdim\Gin@nat@height>\textheight\textheight\else\Gin@nat@height\fi}
\makeatother
% Scale images if necessary, so that they will not overflow the page
% margins by default, and it is still possible to overwrite the defaults
% using explicit options in \includegraphics[width, height, ...]{}
\setkeys{Gin}{width=\maxwidth,height=\maxheight,keepaspectratio}
% Set default figure placement to htbp
\makeatletter
\def\fps@figure{htbp}
\makeatother
\setlength{\emergencystretch}{3em} % prevent overfull lines
\providecommand{\tightlist}{%
  \setlength{\itemsep}{0pt}\setlength{\parskip}{0pt}}
\setcounter{secnumdepth}{-\maxdimen} % remove section numbering
\ifLuaTeX
  \usepackage{selnolig}  % disable illegal ligatures
\fi
\IfFileExists{bookmark.sty}{\usepackage{bookmark}}{\usepackage{hyperref}}
\IfFileExists{xurl.sty}{\usepackage{xurl}}{} % add URL line breaks if available
\urlstyle{same}
\hypersetup{
  pdftitle={Time Series Homework: Univariate \& Multivariate Analysis},
  pdfauthor={Your Names},
  hidelinks,
  pdfcreator={LaTeX via pandoc}}

\title{Time Series Homework: Univariate \& Multivariate Analysis}
\author{Your Names}
\date{2025-04-24}

\begin{document}
\maketitle

{
\setcounter{tocdepth}{3}
\tableofcontents
}
\hypertarget{data-retrieval-and-preparation}{%
\section{Data Retrieval and
Preparation}\label{data-retrieval-and-preparation}}

\begin{Shaded}
\begin{Highlighting}[]
\CommentTok{\# Download from FRED}
\FunctionTok{getSymbols}\NormalTok{(}\FunctionTok{c}\NormalTok{(}\StringTok{"GDPC1"}\NormalTok{,}\StringTok{"UNRATE"}\NormalTok{), }\AttributeTok{src =} \StringTok{"FRED"}\NormalTok{, }\AttributeTok{from =} \StringTok{"1950{-}01{-}01"}\NormalTok{)}
\end{Highlighting}
\end{Shaded}

\begin{verbatim}
## [1] "GDPC1"  "UNRATE"
\end{verbatim}

\begin{Shaded}
\begin{Highlighting}[]
\CommentTok{\# TS objects: quarterly GDP, monthly unemployment}
\NormalTok{gdp     }\OtherTok{\textless{}{-}} \FunctionTok{ts}\NormalTok{(}\FunctionTok{as.numeric}\NormalTok{(GDPC1), }\AttributeTok{start =} \FunctionTok{c}\NormalTok{(}\DecValTok{1947}\NormalTok{,}\DecValTok{1}\NormalTok{), }\AttributeTok{frequency =} \DecValTok{4}\NormalTok{)}
\NormalTok{unemp   }\OtherTok{\textless{}{-}} \FunctionTok{ts}\NormalTok{(}\FunctionTok{as.numeric}\NormalTok{(UNRATE), }\AttributeTok{start =} \FunctionTok{c}\NormalTok{(}\DecValTok{1948}\NormalTok{,}\DecValTok{1}\NormalTok{), }\AttributeTok{frequency =} \DecValTok{12}\NormalTok{)}

\CommentTok{\# Convert monthly to quarterly average}
\NormalTok{unemp\_q }\OtherTok{\textless{}{-}} \FunctionTok{aggregate}\NormalTok{(unemp, }\AttributeTok{nfrequency =} \DecValTok{4}\NormalTok{, }\AttributeTok{FUN =}\NormalTok{ mean)}

\CommentTok{\# Restrict to 1950Q1 onward}
\NormalTok{gdp\_q   }\OtherTok{\textless{}{-}} \FunctionTok{window}\NormalTok{(gdp,   }\AttributeTok{start =} \FunctionTok{c}\NormalTok{(}\DecValTok{1950}\NormalTok{,}\DecValTok{1}\NormalTok{))}
\NormalTok{unemp\_q }\OtherTok{\textless{}{-}} \FunctionTok{window}\NormalTok{(unemp\_q, }\AttributeTok{start =} \FunctionTok{c}\NormalTok{(}\DecValTok{1950}\NormalTok{,}\DecValTok{1}\NormalTok{))}
\end{Highlighting}
\end{Shaded}

\hypertarget{exercise-1-univariate-analysis}{%
\section{Exercise 1: Univariate
Analysis}\label{exercise-1-univariate-analysis}}

\hypertarget{plot-in-levels}{%
\subsection{1. Plot in Levels}\label{plot-in-levels}}

\begin{Shaded}
\begin{Highlighting}[]
\NormalTok{gdp\_plot   }\OtherTok{\textless{}{-}} \FunctionTok{autoplot}\NormalTok{(gdp\_q)   }\SpecialCharTok{+} \FunctionTok{ggtitle}\NormalTok{(}\StringTok{"Real GDP (GDPC1), 1950Q1–2024Q1"}\NormalTok{)}
\NormalTok{unemp\_plot }\OtherTok{\textless{}{-}} \FunctionTok{autoplot}\NormalTok{(unemp\_q) }\SpecialCharTok{+} \FunctionTok{ggtitle}\NormalTok{(}\StringTok{"Unemployment Rate (UNRATE), 1950Q1–2024Q1"}\NormalTok{)}
\FunctionTok{grid.arrange}\NormalTok{(gdp\_plot, unemp\_plot, }\AttributeTok{ncol =} \DecValTok{1}\NormalTok{)}
\end{Highlighting}
\end{Shaded}

\includegraphics{Code_files/figure-latex/plots-levels-1.pdf}

\hypertarget{unit-root-and-stationarity-tests}{%
\subsection{2. Unit-Root and Stationarity
Tests}\label{unit-root-and-stationarity-tests}}

\begin{Shaded}
\begin{Highlighting}[]
\CommentTok{\# GDP: ADF \& KPSS}
\NormalTok{gdp\_adf  }\OtherTok{\textless{}{-}} \FunctionTok{ur.df}\NormalTok{(gdp\_q,  }\AttributeTok{type =} \StringTok{"drift"}\NormalTok{, }\AttributeTok{selectlags =} \StringTok{"AIC"}\NormalTok{)}
\FunctionTok{summary}\NormalTok{(gdp\_adf)}
\end{Highlighting}
\end{Shaded}

\begin{verbatim}
## 
## ############################################### 
## # Augmented Dickey-Fuller Test Unit Root Test # 
## ############################################### 
## 
## Test regression drift 
## 
## 
## Call:
## lm(formula = z.diff ~ z.lag.1 + 1 + z.diff.lag)
## 
## Residuals:
##      Min       1Q   Median       3Q      Max 
## -1816.49   -34.15     5.86    47.43  1117.90 
## 
## Coefficients:
##              Estimate Std. Error t value Pr(>|t|)    
## (Intercept) 25.622333  18.668595   1.372 0.171000    
## z.lag.1      0.005320   0.001544   3.447 0.000654 ***
## z.diff.lag  -0.151086   0.059054  -2.558 0.011035 *  
## ---
## Signif. codes:  0 '***' 0.001 '**' 0.01 '*' 0.05 '.' 0.1 ' ' 1
## 
## Residual standard error: 153.1 on 283 degrees of freedom
## Multiple R-squared:  0.05206,    Adjusted R-squared:  0.04537 
## F-statistic: 7.772 on 2 and 283 DF,  p-value: 0.0005178
## 
## 
## Value of test-statistic is: 3.4467 38.2646 
## 
## Critical values for test statistics: 
##       1pct  5pct 10pct
## tau2 -3.44 -2.87 -2.57
## phi1  6.47  4.61  3.79
\end{verbatim}

\begin{Shaded}
\begin{Highlighting}[]
\NormalTok{gdp\_kpss }\OtherTok{\textless{}{-}} \FunctionTok{ur.kpss}\NormalTok{(gdp\_q, }\AttributeTok{type =} \StringTok{"mu"}\NormalTok{)}
\FunctionTok{summary}\NormalTok{(gdp\_kpss)}
\end{Highlighting}
\end{Shaded}

\begin{verbatim}
## 
## ####################### 
## # KPSS Unit Root Test # 
## ####################### 
## 
## Test is of type: mu with 5 lags. 
## 
## Value of test-statistic is: 4.812 
## 
## Critical value for a significance level of: 
##                 10pct  5pct 2.5pct  1pct
## critical values 0.347 0.463  0.574 0.739
\end{verbatim}

\begin{Shaded}
\begin{Highlighting}[]
\CommentTok{\# Unemployment: ADF \& KPSS}
\NormalTok{un\_adf   }\OtherTok{\textless{}{-}} \FunctionTok{ur.df}\NormalTok{(unemp\_q, }\AttributeTok{type =} \StringTok{"drift"}\NormalTok{, }\AttributeTok{selectlags =} \StringTok{"AIC"}\NormalTok{)}
\FunctionTok{summary}\NormalTok{(un\_adf)}
\end{Highlighting}
\end{Shaded}

\begin{verbatim}
## 
## ############################################### 
## # Augmented Dickey-Fuller Test Unit Root Test # 
## ############################################### 
## 
## Test regression drift 
## 
## 
## Call:
## lm(formula = z.diff ~ z.lag.1 + 1 + z.diff.lag)
## 
## Residuals:
##     Min      1Q  Median      3Q     Max 
## -3.7980 -0.2218 -0.0994  0.1225  8.9807 
## 
## Coefficients:
##             Estimate Std. Error t value Pr(>|t|)    
## (Intercept)  0.53008    0.14590   3.633 0.000331 ***
## z.lag.1     -0.09147    0.02437  -3.753 0.000211 ***
## z.diff.lag   0.02804    0.05872   0.478 0.633320    
## ---
## Signif. codes:  0 '***' 0.001 '**' 0.01 '*' 0.05 '.' 0.1 ' ' 1
## 
## Residual standard error: 0.69 on 288 degrees of freedom
## Multiple R-squared:  0.04688,    Adjusted R-squared:  0.04026 
## F-statistic: 7.083 on 2 and 288 DF,  p-value: 0.0009933
## 
## 
## Value of test-statistic is: -3.7532 7.0482 
## 
## Critical values for test statistics: 
##       1pct  5pct 10pct
## tau2 -3.44 -2.87 -2.57
## phi1  6.47  4.61  3.79
\end{verbatim}

\begin{Shaded}
\begin{Highlighting}[]
\NormalTok{un\_kpss  }\OtherTok{\textless{}{-}} \FunctionTok{ur.kpss}\NormalTok{(unemp\_q, }\AttributeTok{type =} \StringTok{"mu"}\NormalTok{)}
\FunctionTok{summary}\NormalTok{(un\_kpss)}
\end{Highlighting}
\end{Shaded}

\begin{verbatim}
## 
## ####################### 
## # KPSS Unit Root Test # 
## ####################### 
## 
## Test is of type: mu with 5 lags. 
## 
## Value of test-statistic is: 0.338 
## 
## Critical value for a significance level of: 
##                 10pct  5pct 2.5pct  1pct
## critical values 0.347 0.463  0.574 0.739
\end{verbatim}

\hypertarget{arparmapq-identification-selection}{%
\subsection{3. AR(p)/ARMA(p,q) Identification \&
Selection}\label{arparmapq-identification-selection}}

\begin{Shaded}
\begin{Highlighting}[]
\CommentTok{\# Difference GDP}
\NormalTok{dgdp }\OtherTok{\textless{}{-}} \FunctionTok{diff}\NormalTok{(gdp\_q)}

\CommentTok{\# ACF/PACF plots}
\NormalTok{p\_acf\_dgdp }\OtherTok{\textless{}{-}} \FunctionTok{autoplot}\NormalTok{(}\FunctionTok{Acf}\NormalTok{(dgdp,  }\AttributeTok{lag.max =} \DecValTok{20}\NormalTok{)) }\SpecialCharTok{+} \FunctionTok{ggtitle}\NormalTok{(}\StringTok{"ACF of ∆GDP"}\NormalTok{)}
\end{Highlighting}
\end{Shaded}

\includegraphics{Code_files/figure-latex/acf-pacf-1.pdf}

\begin{Shaded}
\begin{Highlighting}[]
\NormalTok{p\_pacf\_dgdp }\OtherTok{\textless{}{-}} \FunctionTok{autoplot}\NormalTok{(}\FunctionTok{Pacf}\NormalTok{(dgdp, }\AttributeTok{lag.max =} \DecValTok{20}\NormalTok{)) }\SpecialCharTok{+} \FunctionTok{ggtitle}\NormalTok{(}\StringTok{"PACF of ∆GDP"}\NormalTok{)}
\end{Highlighting}
\end{Shaded}

\includegraphics{Code_files/figure-latex/acf-pacf-2.pdf}

\begin{Shaded}
\begin{Highlighting}[]
\NormalTok{p\_acf\_un   }\OtherTok{\textless{}{-}} \FunctionTok{autoplot}\NormalTok{(}\FunctionTok{Acf}\NormalTok{(unemp\_q, }\AttributeTok{lag.max =} \DecValTok{20}\NormalTok{)) }\SpecialCharTok{+} \FunctionTok{ggtitle}\NormalTok{(}\StringTok{"ACF of UNRATE"}\NormalTok{)}
\end{Highlighting}
\end{Shaded}

\includegraphics{Code_files/figure-latex/acf-pacf-3.pdf}

\begin{Shaded}
\begin{Highlighting}[]
\NormalTok{p\_pacf\_un  }\OtherTok{\textless{}{-}} \FunctionTok{autoplot}\NormalTok{(}\FunctionTok{Pacf}\NormalTok{(unemp\_q, }\AttributeTok{lag.max =} \DecValTok{20}\NormalTok{)) }\SpecialCharTok{+} \FunctionTok{ggtitle}\NormalTok{(}\StringTok{"PACF of UNRATE"}\NormalTok{)}
\end{Highlighting}
\end{Shaded}

\includegraphics{Code_files/figure-latex/acf-pacf-4.pdf}

\begin{Shaded}
\begin{Highlighting}[]
\FunctionTok{grid.arrange}\NormalTok{(p\_acf\_dgdp, p\_pacf\_dgdp, p\_acf\_un, p\_pacf\_un, }\AttributeTok{ncol =} \DecValTok{2}\NormalTok{)}
\end{Highlighting}
\end{Shaded}

\includegraphics{Code_files/figure-latex/acf-pacf-5.pdf}

\begin{Shaded}
\begin{Highlighting}[]
\CommentTok{\# Fit ARIMA models for GDP using ML to avoid CSS AR non{-}stationarity error}
\NormalTok{fit\_g1 }\OtherTok{\textless{}{-}} \FunctionTok{Arima}\NormalTok{(gdp\_q, }\AttributeTok{order =} \FunctionTok{c}\NormalTok{(}\DecValTok{1}\NormalTok{,}\DecValTok{1}\NormalTok{,}\DecValTok{0}\NormalTok{), }\AttributeTok{method =} \StringTok{"ML"}\NormalTok{)}
\NormalTok{fit\_g2 }\OtherTok{\textless{}{-}} \FunctionTok{Arima}\NormalTok{(gdp\_q, }\AttributeTok{order =} \FunctionTok{c}\NormalTok{(}\DecValTok{1}\NormalTok{,}\DecValTok{1}\NormalTok{,}\DecValTok{1}\NormalTok{), }\AttributeTok{method =} \StringTok{"ML"}\NormalTok{)}
\NormalTok{fit\_g3 }\OtherTok{\textless{}{-}} \FunctionTok{Arima}\NormalTok{(gdp\_q, }\AttributeTok{order =} \FunctionTok{c}\NormalTok{(}\DecValTok{2}\NormalTok{,}\DecValTok{1}\NormalTok{,}\DecValTok{0}\NormalTok{), }\AttributeTok{method =} \StringTok{"ML"}\NormalTok{)}

\CommentTok{\# Compare information criteria}
\FunctionTok{AIC}\NormalTok{(fit\_g1, fit\_g2, fit\_g3)}
\end{Highlighting}
\end{Shaded}

\begin{verbatim}
##        df      AIC
## fit_g1  2 3770.819
## fit_g2  3 3719.902
## fit_g3  3 3764.447
\end{verbatim}

\begin{Shaded}
\begin{Highlighting}[]
\FunctionTok{BIC}\NormalTok{(fit\_g1, fit\_g2, fit\_g3)}
\end{Highlighting}
\end{Shaded}

\begin{verbatim}
##        df      BIC
## fit_g1  2 3778.138
## fit_g2  3 3730.881
## fit_g3  3 3775.425
\end{verbatim}

\begin{Shaded}
\begin{Highlighting}[]
\CommentTok{\# Fit ARMA models for unemployment (stationary in levels)}
\NormalTok{fit\_u1 }\OtherTok{\textless{}{-}} \FunctionTok{Arima}\NormalTok{(unemp\_q, }\AttributeTok{order =} \FunctionTok{c}\NormalTok{(}\DecValTok{1}\NormalTok{,}\DecValTok{0}\NormalTok{,}\DecValTok{0}\NormalTok{), }\AttributeTok{method =} \StringTok{"ML"}\NormalTok{)}
\NormalTok{fit\_u2 }\OtherTok{\textless{}{-}} \FunctionTok{Arima}\NormalTok{(unemp\_q, }\AttributeTok{order =} \FunctionTok{c}\NormalTok{(}\DecValTok{1}\NormalTok{,}\DecValTok{0}\NormalTok{,}\DecValTok{1}\NormalTok{), }\AttributeTok{method =} \StringTok{"ML"}\NormalTok{)}
\FunctionTok{AIC}\NormalTok{(fit\_u1, fit\_u2)}
\end{Highlighting}
\end{Shaded}

\begin{verbatim}
##        df      AIC
## fit_u1  3 619.6685
## fit_u2  4 621.4511
\end{verbatim}

\begin{Shaded}
\begin{Highlighting}[]
\FunctionTok{BIC}\NormalTok{(fit\_u1, fit\_u2)}
\end{Highlighting}
\end{Shaded}

\begin{verbatim}
##        df      BIC
## fit_u1  3 630.7090
## fit_u2  4 636.1718
\end{verbatim}

\hypertarget{forecasts-for-unrate}{%
\subsection{4. Forecasts for UNRATE}\label{forecasts-for-unrate}}

\begin{Shaded}
\begin{Highlighting}[]
\CommentTok{\# Train/Test split}
\NormalTok{train\_u     }\OtherTok{\textless{}{-}} \FunctionTok{window}\NormalTok{(unemp\_q, }\AttributeTok{end =} \FunctionTok{c}\NormalTok{(}\DecValTok{2022}\NormalTok{,}\DecValTok{4}\NormalTok{))}
\NormalTok{test\_u      }\OtherTok{\textless{}{-}} \FunctionTok{window}\NormalTok{(unemp\_q, }\AttributeTok{start =} \FunctionTok{c}\NormalTok{(}\DecValTok{2023}\NormalTok{,}\DecValTok{1}\NormalTok{))}
\NormalTok{fit\_train\_u }\OtherTok{\textless{}{-}} \FunctionTok{Arima}\NormalTok{(train\_u, }\AttributeTok{order =} \FunctionTok{c}\NormalTok{(}\DecValTok{1}\NormalTok{,}\DecValTok{0}\NormalTok{,}\DecValTok{0}\NormalTok{), }\AttributeTok{method =} \StringTok{"ML"}\NormalTok{)}

\CommentTok{\# Forecast}
\NormalTok{fc\_u }\OtherTok{\textless{}{-}} \FunctionTok{forecast}\NormalTok{(fit\_train\_u, }\AttributeTok{h =} \FunctionTok{length}\NormalTok{(test\_u))}
\FunctionTok{autoplot}\NormalTok{(fit\_train\_u) }\SpecialCharTok{+}
  \FunctionTok{autolayer}\NormalTok{(fc\_u, }\AttributeTok{series =} \StringTok{"Forecast"}\NormalTok{, }\AttributeTok{PI =} \ConstantTok{TRUE}\NormalTok{) }\SpecialCharTok{+}
  \FunctionTok{autolayer}\NormalTok{(test\_u, }\AttributeTok{series =} \StringTok{"Actual"}\NormalTok{) }\SpecialCharTok{+}
  \FunctionTok{ggtitle}\NormalTok{(}\StringTok{"UNRATE: In{-}Sample \& Out{-}of{-}Sample Forecasts"}\NormalTok{)}
\end{Highlighting}
\end{Shaded}

\includegraphics{Code_files/figure-latex/forecast-unemp-1.pdf}

\hypertarget{exercise-2-multivariate-var-analysis}{%
\section{Exercise 2: Multivariate (VAR)
Analysis}\label{exercise-2-multivariate-var-analysis}}

\hypertarget{lag-length-selection}{%
\subsection{1. Lag-Length Selection}\label{lag-length-selection}}

\begin{Shaded}
\begin{Highlighting}[]
\CommentTok{\# Align both series on common time window via ts.intersect}
\NormalTok{ts\_data    }\OtherTok{\textless{}{-}} \FunctionTok{ts.intersect}\NormalTok{(dgdp, unemp\_q)}
\FunctionTok{colnames}\NormalTok{(ts\_data) }\OtherTok{\textless{}{-}} \FunctionTok{c}\NormalTok{(}\StringTok{"dGDP"}\NormalTok{,}\StringTok{"UNR"}\NormalTok{)}
\CommentTok{\# ts\_data runs from max(start(dgdp), start(unemp\_q)) with no NAs}
\FunctionTok{VARselect}\NormalTok{(ts\_data, }\AttributeTok{lag.max =} \DecValTok{8}\NormalTok{, }\AttributeTok{type =} \StringTok{"const"}\NormalTok{)}
\end{Highlighting}
\end{Shaded}

\begin{verbatim}
## $selection
## AIC(n)  HQ(n)  SC(n) FPE(n) 
##      7      4      4      7 
## 
## $criteria
##                   1            2            3           4           5
## AIC(n)     9.408797     9.435527     9.351371    7.829751    7.835960
## HQ(n)      9.440122     9.487736     9.424464    7.923729    7.950821
## SC(n)      9.486887     9.565678     9.533582    8.064023    8.122292
## FPE(n) 12195.206950 12525.655674 11514.836648 2514.416788 2530.168862
##                  6           7           8
## AIC(n)    7.825692    7.818331    7.823286
## HQ(n)     7.961437    7.974960    8.000799
## SC(n)     8.164084    8.208784    8.265799
## FPE(n) 2504.455655 2486.270286 2498.855428
\end{verbatim}

\hypertarget{estimate-var2-diagnostics}{%
\subsection{2. Estimate VAR(2) \&
Diagnostics}\label{estimate-var2-diagnostics}}

\begin{Shaded}
\begin{Highlighting}[]
\NormalTok{var2 }\OtherTok{\textless{}{-}} \FunctionTok{VAR}\NormalTok{(ts\_data, }\AttributeTok{p =} \DecValTok{2}\NormalTok{, }\AttributeTok{type =} \StringTok{"const"}\NormalTok{)}
\FunctionTok{summary}\NormalTok{(var2)}
\end{Highlighting}
\end{Shaded}

\begin{verbatim}
## 
## VAR Estimation Results:
## ========================= 
## Endogenous variables: dGDP, UNR 
## Deterministic variables: const 
## Sample size: 285 
## Log Likelihood: -2140.632 
## Roots of the characteristic polynomial:
## 0.9044 0.1505 0.1505 0.02604
## Call:
## VAR(y = ts_data, p = 2, type = "const")
## 
## 
## Estimation results for equation dGDP: 
## ===================================== 
## dGDP = dGDP.l1 + UNR.l1 + dGDP.l2 + UNR.l2 + const 
## 
##         Estimate Std. Error t value Pr(>|t|)  
## dGDP.l1 -0.12216    0.05990  -2.040   0.0423 *
## UNR.l1   1.83634   13.38552   0.137   0.8910  
## dGDP.l2 -0.02214    0.05997  -0.369   0.7122  
## UNR.l2   7.14925   13.32087   0.537   0.5919  
## const   30.73055   33.79617   0.909   0.3640  
## ---
## Signif. codes:  0 '***' 0.001 '**' 0.01 '*' 0.05 '.' 0.1 ' ' 1
## 
## 
## Residual standard error: 156.2 on 280 degrees of freedom
## Multiple R-Squared: 0.02149, Adjusted R-squared: 0.007507 
## F-statistic: 1.537 on 4 and 280 DF,  p-value: 0.1916 
## 
## 
## Estimation results for equation UNR: 
## ==================================== 
## UNR = dGDP.l1 + UNR.l1 + dGDP.l2 + UNR.l2 + const 
## 
##           Estimate Std. Error t value Pr(>|t|)    
## dGDP.l1  1.403e-04  2.680e-04   0.523 0.601043    
## UNR.l1   9.340e-01  5.990e-02  15.593  < 2e-16 ***
## dGDP.l2  1.234e-05  2.684e-04   0.046 0.963345    
## UNR.l2  -2.807e-02  5.961e-02  -0.471 0.638107    
## const    5.359e-01  1.512e-01   3.544 0.000462 ***
## ---
## Signif. codes:  0 '***' 0.001 '**' 0.01 '*' 0.05 '.' 0.1 ' ' 1
## 
## 
## Residual standard error: 0.6992 on 280 degrees of freedom
## Multiple R-Squared: 0.8307,  Adjusted R-squared: 0.8283 
## F-statistic: 343.5 on 4 and 280 DF,  p-value: < 2.2e-16 
## 
## 
## 
## Covariance matrix of residuals:
##           dGDP    UNR
## dGDP 24411.608 8.0735
## UNR      8.073 0.4888
## 
## Correlation matrix of residuals:
##         dGDP     UNR
## dGDP 1.00000 0.07391
## UNR  0.07391 1.00000
\end{verbatim}

\begin{Shaded}
\begin{Highlighting}[]
\FunctionTok{serial.test}\NormalTok{(var2, }\AttributeTok{lags.pt =} \DecValTok{16}\NormalTok{, }\AttributeTok{type =} \StringTok{"PT.adjusted"}\NormalTok{)}
\end{Highlighting}
\end{Shaded}

\begin{verbatim}
## 
##  Portmanteau Test (adjusted)
## 
## data:  Residuals of VAR object var2
## Chi-squared = 243.58, df = 56, p-value < 2.2e-16
\end{verbatim}

\begin{Shaded}
\begin{Highlighting}[]
\FunctionTok{arch.test}\NormalTok{(var2, }\AttributeTok{lags.multi =} \DecValTok{5}\NormalTok{)}
\end{Highlighting}
\end{Shaded}

\begin{verbatim}
## 
##  ARCH (multivariate)
## 
## data:  Residuals of VAR object var2
## Chi-squared = 603.65, df = 45, p-value < 2.2e-16
\end{verbatim}

\begin{Shaded}
\begin{Highlighting}[]
\FunctionTok{normality.test}\NormalTok{(var2)}
\end{Highlighting}
\end{Shaded}

\begin{verbatim}
## $JB
## 
##  JB-Test (multivariate)
## 
## data:  Residuals of VAR object var2
## Chi-squared = 162506, df = 4, p-value < 2.2e-16
## 
## 
## $Skewness
## 
##  Skewness only (multivariate)
## 
## data:  Residuals of VAR object var2
## Chi-squared = 2751.7, df = 2, p-value < 2.2e-16
## 
## 
## $Kurtosis
## 
##  Kurtosis only (multivariate)
## 
## data:  Residuals of VAR object var2
## Chi-squared = 159754, df = 2, p-value < 2.2e-16
\end{verbatim}

\hypertarget{forecast-gdp}{%
\subsection{3. Forecast ∆GDP}\label{forecast-gdp}}

\begin{Shaded}
\begin{Highlighting}[]
\NormalTok{n\_ahead }\OtherTok{\textless{}{-}} \DecValTok{8}
\NormalTok{fc\_var  }\OtherTok{\textless{}{-}} \FunctionTok{predict}\NormalTok{(var2, }\AttributeTok{n.ahead =}\NormalTok{ n\_ahead, }\AttributeTok{ci =} \FloatTok{0.95}\NormalTok{)}
\FunctionTok{plot}\NormalTok{(fc\_var, }\AttributeTok{names =} \StringTok{"dGDP"}\NormalTok{)}
\end{Highlighting}
\end{Shaded}

\includegraphics{Code_files/figure-latex/var-forecast-1.pdf}

\hypertarget{cholesky-ordering-orthogonalized-irfs}{%
\subsection{4--5. Cholesky Ordering \& Orthogonalized
IRFs}\label{cholesky-ordering-orthogonalized-irfs}}

\begin{Shaded}
\begin{Highlighting}[]
\NormalTok{irf\_o }\OtherTok{\textless{}{-}} \FunctionTok{irf}\NormalTok{(}
\NormalTok{  var2,}
  \AttributeTok{impulse  =} \StringTok{"dGDP"}\NormalTok{,}
  \AttributeTok{response =} \FunctionTok{c}\NormalTok{(}\StringTok{"dGDP"}\NormalTok{,}\StringTok{"UNR"}\NormalTok{),}
  \AttributeTok{n.ahead  =} \DecValTok{20}\NormalTok{,}
  \AttributeTok{ortho    =} \ConstantTok{TRUE}\NormalTok{,}
  \AttributeTok{boot     =} \ConstantTok{TRUE}
\NormalTok{)}
\FunctionTok{plot}\NormalTok{(irf\_o)}
\end{Highlighting}
\end{Shaded}

\includegraphics{Code_files/figure-latex/irf-orth-1.pdf}

\hypertarget{reverse-ordering-comparison}{%
\subsection{6. Reverse Ordering \&
Comparison}\label{reverse-ordering-comparison}}

\begin{Shaded}
\begin{Highlighting}[]
\CommentTok{\# Align and swap ordering}
\NormalTok{ts\_data\_rev }\OtherTok{\textless{}{-}} \FunctionTok{ts.intersect}\NormalTok{(unemp\_q, dgdp)}
\FunctionTok{colnames}\NormalTok{(ts\_data\_rev) }\OtherTok{\textless{}{-}} \FunctionTok{c}\NormalTok{(}\StringTok{"UNR"}\NormalTok{,}\StringTok{"dGDP"}\NormalTok{)}
\NormalTok{var\_rev }\OtherTok{\textless{}{-}} \FunctionTok{VAR}\NormalTok{(ts\_data\_rev, }\AttributeTok{p =} \DecValTok{2}\NormalTok{, }\AttributeTok{type =} \StringTok{"const"}\NormalTok{)}
\NormalTok{irf\_rev }\OtherTok{\textless{}{-}} \FunctionTok{irf}\NormalTok{(}
\NormalTok{  var\_rev,}
  \AttributeTok{impulse  =} \StringTok{"UNR"}\NormalTok{,}
  \AttributeTok{response =} \FunctionTok{c}\NormalTok{(}\StringTok{"UNR"}\NormalTok{,}\StringTok{"dGDP"}\NormalTok{),}
  \AttributeTok{n.ahead  =} \DecValTok{20}\NormalTok{,}
  \AttributeTok{ortho    =} \ConstantTok{TRUE}\NormalTok{,}
  \AttributeTok{boot     =} \ConstantTok{TRUE}
\NormalTok{)}
\FunctionTok{plot}\NormalTok{(irf\_rev)}
\end{Highlighting}
\end{Shaded}

\includegraphics{Code_files/figure-latex/irf-reverse-1.pdf}

\textbf{End of Document}

\end{document}
